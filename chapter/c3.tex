
\chapter{算法分析及系统架构设计:以回声状态网络为例}

\section{基于投影的模型压缩算法}
\subsection{高速回声状态网络结构}
\subsection{本征正交分解与离散经验插值}
介绍高速回声状态网络的生成,算法顺序介绍,or硬件需求倒推
\subsection{压缩算法的评估与分析}
从精度和复杂度方面分析。

\section{系统整体架构设计}
应用场景介绍引入软硬件功能划分。
\subsection{前向传播及压缩流程}
\subsection{软硬件功能划分}
\subsection{系统整体架构}

\section{面向算法的定制化硬件分析}
\subsection{算法需求分析}
1需要实现原始网络,高速网络两个网络结构。
2压缩网络结构尺寸可调节。
3模型压缩过程。
\subsection{硬件模块划分与复用性分析}
\subsection{激活函数分段近似}
\subsection{计算资源与存储资源需求分析}

\section{基于FPGA的加速器设计}
\subsection{硬件加速器整体架构}

\subsection{存储架构设计}

\subsection{计算架构设计}

\subsection{矩阵向量乘法模块}

\subsection{激活函数模块}

\subsection{IP核互联设计}

\section{本章小结}
