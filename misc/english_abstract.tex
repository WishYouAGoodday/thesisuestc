
\begin{englishabstract}
Recurrent Neural Networks (RNNs) are a type of neural network designed specifically for processing sequential data, 
and are widely used in various fields such as speech recognition, machine translation, and dynamic system modeling. 
RNNs have shown superior performance compared to other neural networks in time sequences related tasks.
With the increasing complexity of tasks and demanding for better model prediction accuracy, the parameter size of 
Recurrent Neural Networks (RNNs) has also become larger, which resulting 
in significant storage and computational pressure on hardware implementation platforms and also lead to the high latency problem. 
Those problems hinder the wider application of RNNs in various scenarios, such as embedded system and IoT environments.
Existing work has proposed some classical solutions, such as pruning algorithms and hardware accelerators, focusing on 
model compression and hardware acceleration techniques. 
However, these solutions have great shortcomings such as high compression costs and strong accelerator specialization, 
making them unsuitable for scenarios requiring dynamic adjustment of accuracy and speed, which are commonly encountered. 
Therefore, there is significant practical value in developing Recurrent Neural Network (RNN) acceleration technologies with 
the ability to dynamically adjust precision and speed.

To solve the above problems, this thesis researches the acceleration techniques for the forward propagation process of recurrent neural networks, 
designs and implements an RNN acceleration system with adjustable precision and 
speed based on FPGA. 
This system leverages the low-cost advantage of projection-based compression algorithm and organically integrates it with 
the forward propagation process of the network to generate and switch to specified network sizes during system running,
ultimately achieving the goal of adjusting the system's accuracy and speed. 
Firstly, this thesis analyzed and designed the system architecture, and mapped each function component to specific software 
and hardware implementations. The rational partitioning enables the system to operate efficiently.
Secondly, in terms of software algorithm design, this thesis considers possible emergent situations during system operation 
and proposes methods based on pre-set projection matrices method and state sampling method, respectively corresponding to 
normal state scenarios and abnormal state scenarios. 
Sufficient consideration of various scenarios enhances the system's robustness in different environments.
Thirdly, in terms of hardware implementation, this thesis designed a hardware accelerator to accelerate the forward propagation 
process of recurrent neural networks. The accelerator can run two different network models and adjust the model's size, 
Dynamic adjustability is achieved from the blocked matrix-vector multiplication.
Finally, this thesis optimizes the resource consumption of the system by using a segmented cubic function approximation method to 
optimize the resource consumption of the activation function module.
Experimental results on the system performance show that the designed and implemented recurrent neural network acceleration system 
in this thesis has dynamic adjustability of accuracy and speed. Performance testing experiments on the accelerator indicate that
the resource consumption of this accelerator is reasonable.

\englishkeyword{Recurrent Neural NetWorks, Echo State Networks, Hardware Accelerator, FPGA, High-Level Synthesis}
\end{englishabstract}
